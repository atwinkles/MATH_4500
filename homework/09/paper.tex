\documentclass{article}
\usepackage[utf8]{inputenc}
\usepackage{amsmath}
\usepackage{amsthm}
\usepackage{amsfonts}
\usepackage{amssymb}
\usepackage{amstext}
\usepackage{gensymb}
\usepackage{graphicx}
\usepackage{enumerate}
\pagenumbering{arabic}
\usepackage{fancyhdr}
\usepackage[margin=0.75in]{geometry}
\usepackage{eucal}
\usepackage{parskip} % removes auto indentation for paragraphs
\usepackage{enumitem} % changes the indexing for enumerate
\setlist[enumerate,1]{label = {(\alph*)}}

\def\N{\mathbb{N}}
\def\Z{\mathbb{Z}}
\def\Q{\mathbb{Q}}
\def\R{\mathbb{R}}
\newcommand{\Mod}[1]{\ (\text{mod}\ #1)}
\newcommand{\Problem}[1]{\textbf{Problem #1}}
\newcommand{\li}[0]{\liminf_{n\to\infty}}
\newcommand{\ls}[0]{\limsup_{n\to\infty}}
\newcommand{\dl}[2]{\displaystyle\lim_{#1 \to #2}}

\linespread{1.5}

\pagestyle{fancy}
\fancyhf{}
\rhead{MATH 4500}
\lhead{Alexander Winkles}
\chead{\Large \textbf{Problem Set 9}}
\cfoot{Page \thepage}

\begin{document}

\Problem{7.3.21}

Consider the numerical integration rule $\displaystyle\int_{-1}^1 f(x)\ dx \approx Af\left(-\sqrt{\frac{3}{5}}\right) + Bf(0) + Cf\left(\sqrt{\frac{3}{5}}\right)$.
\begin{enumerate}
\item The following is the linear system that must be solved to determine $A,B,$ and $C$: 
\begin{align*}
\displaystyle\int_{-1}^1 dx = 2 &= A + B + C\\
\displaystyle\int_{-1}^1 x\ dx = 0 &= -\sqrt{\frac{3}{5}}A + \sqrt{\frac{3}{5}}C\\
\displaystyle\int_{-1}^1 x^2\ dx = \frac{2}{3} &= \frac{3}{5}A + \frac{3}{5}C
\end{align*}
Solving this system in MATLAB with "linsolve" gives 
\begin{align*}
A &= 0.5555556 = \frac{5}{9}\\
B &= 0.8888889 = \frac{8}{9}\\
C &= 0.5555556 = \frac{5}{9}	
\end{align*}
\item The following are the integrals that must be evaluated to find $A,B,$ and $C$ using Newton-Cotes:
\begin{align*}
\displaystyle\int_{-1}^1 \frac{(x-0)(x-	\sqrt{\frac{3}{5}})}{(-\sqrt{\frac{3}{5}} -0)(-\sqrt{\frac{3}{5}} - \sqrt{\frac{3}{5}})}\ dx &= 0.5555556 = \frac{5}{9}\\
\displaystyle\int_{-1}^1 \frac{(x+\sqrt{\frac{3}{5}})(x-\sqrt{\frac{3}{5}})}{(-\sqrt{\frac{3}{5}})(\sqrt{\frac{3}{5}})}\ dx &= 0.8888889 = \frac{8}{9}\\
\displaystyle\int_{-1}^1 \frac{(x+\sqrt{\frac{3}{5}})(x-0)}{\sqrt{\frac{3}{5}}+\sqrt{\frac{3}{5}})(\sqrt{\frac{3}{5}}-0)}\ dx &= 0.5555556 = \frac{5}{9}
\end{align*}

\end{enumerate}

\Problem{7.3.22}

Let $\displaystyle\int_{-1}^1 f(x)\ dx \approx \frac{5}{9}f\left(-\sqrt{\frac{3}{5}}\right) + \frac{8}{9}f(0)+\frac{5}{9}f\left(\sqrt{\frac{3}{5}}\right)$. 
\begin{enumerate}
\item We wish to solve $\displaystyle\int_0^{\pi/2}x\ dx$ using the above formula. Using the change of intervals formula, our given Gaussian quadrature becomes $	\frac{\pi}{4} \left(\frac{5}{9}(\frac{\pi}{4}\cdot (-\sqrt{\frac{3}{5}})  +\frac{\pi}{4}) + \frac{8}{9}(\frac{\pi}{4}) + \frac{5}{9}(\frac{\pi}{4}\cdot\sqrt{\frac{3}{5}}  +\frac{\pi}{4})\right) = 1.23370$.
\item We wish to solve $\displaystyle\int_0^4 \frac{\sin{t}}{t}\ dt$ using the above formula. Using the change of interval formula, our given Gaussian quadrature becomes $2\left(\frac{5}{9}\frac{\sin{(2\cdot (-\sqrt{\frac{3}{5}}) + 2)}}{2\cdot (-\sqrt{\frac{3}{5}})} + \frac{8}{9}\frac{\sin{2}}{2} + \frac{5}{9}\frac{\sin{(2\cdot (\sqrt{\frac{3}{5}}) + 2)}}{2\cdot (\sqrt{\frac{3}{5}})} \right) = 1.75802$.
\end{enumerate}

\Problem{7.4.4}

Recall Simpson's rule, which states $\displaystyle\int_a^b f(x)\ dx \approx \frac{b-a}{6}\left[f(a) + 4f\left(\frac{a+b}{2}\right) + f(b)\right]$. The second column of the Romberg array takes the form of $R(n,1) = R(n,0) + \frac{1}{4^m-1}[R(n,0) - R(0,0)] = $

\Problem{7.4.5}

We wish to show by induction that $I - R(n,m-1) = a_1h^{2m} + a_2h^{2m+2} + a_3h^{2m+4} + ...$. Let our base case be when $m = 1$, then we have $I - R(n,0) = a_1h^2 + a_2h^4 + ...$, so the base case holds. Now for our inductive step, suppose for $m \in \N$ that $I - R(n,m-1) = a_1h^{2m} + a_2h^{2m+2} + a_3h^{2m+4} + ... \forall n \in \N$. Recall that $h_n = 2h_{n-1}$.
\begin{align*}
	\Rightarrow I - \frac{4^mR(n,m-1)-R(n-1,m-1)}{4^m-1} &= \frac{4^m(I - R(n,m-1))}{4^m-1} - \frac{I - R(n-1,m-1)}{4^m-1}\\
	&= \frac{4^m(a_1h_{n+1}^{2m} + a_2h_{n+1}^{2m+2} + ...)}{4^m-1} - \frac{a_1h_{n}^{2m} + a_2h_{n}^{2m} + ...}{4^m-1}\\
	&= \frac{4^m(a_1(\frac{h_{n}}{2})^{2m} + a_2(\frac{h_{n}}{2})^{2m+2} + ...) - (a_1h_{n}^{2m} + a_2h_{n}^{2m+2}+ ...)}{4^m-1}\\
	&=b_1h_{n+1}^{2m+2} + b_2h_{n+1}^{2m+4} + ...
\end{align*}
as desired.

\Problem{7.4.6}

We wish to apply the Romberg algorithm to find $R(2,2)$ for the following integrals. For this, I wrote a MATLAB code called "romberg.m". 
\begin{enumerate}
\item $\displaystyle\int_1^3 \frac{dx}{x}$. We now construct the Romberg array:

\begin{center}
\begin{tabular}{cccc}
	$n \backslash m$ & 0 & 1 & 2\\
	0 & 1.333 & 0 & 0\\
	1 & 1.667 & 1.1111 & 0\\
	2 & 1.1167 & 1.1000 & 1.0993
\end{tabular}
\end{center}
Thus, $R(2,2) = 1.0993$.

\item $\displaystyle\int_0^{\pi/2} (\frac{x}{\pi})^2\ dx$ in terms of $\pi$. As before, we construct the Romberg array using MATLAB:

\begin{center}
\begin{tabular}{cccc}
	$n \backslash m$ & 0 & 1 & 2\\
	0 & 0.1936 & 0 & 0\\
	1 & 0.1473 & 0.1309 & 0\\
	2 & 0.1350 & 0.1309 & 0.1309
\end{tabular}
\end{center}

In terms of $\pi$, this is approximately 

\begin{center}
\begin{tabular}{cccc}
	$n \backslash m$ & 0 & 1 & 2\\
	0 & $\frac{\pi}{16}$ & 0 & 0\\
	1 & $\frac{3\pi}{64}$ & $\frac{\pi}{24}$ & 0\\
	2 & $\frac{11\pi}{256}$ & $\frac{\pi}{24}$ & $\frac{\pi}{24}$
\end{tabular}
\end{center}

Thus, $R(2,2) = \frac{\pi}{24}$. 

\end{enumerate}


\end{document}
