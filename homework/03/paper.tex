\documentclass{article}
\usepackage[utf8]{inputenc}
\usepackage{amsmath}
\usepackage{amsthm}
\usepackage{amsfonts}
\usepackage{amssymb}
\usepackage{amstext}
\usepackage{gensymb}
\usepackage{graphicx}
%\usepackage{bbold}
%\usepackage{url}
%\usepackage{booktabs}
%\usepackage{marvosym}
%\usepackage{wasysym}
\pagenumbering{arabic}
\usepackage{fancyhdr}
\usepackage[margin=1.0in]{geometry}
\usepackage{eucal}
\def\N{\mathbb{N}}
\def\Z{\mathbb{Z}}
\def\Q{\mathbb{Q}}
\def\R{\mathbb{R}}
\newcommand{\Mod}[1]{\ (\text{mod}\ #1)}

\pagestyle{fancy}
\fancyhf{}
\rhead{MATH 4500}
\lhead{Alexander Winkles}
\chead{\Large \textbf{Problem Set 3}}
\cfoot{Page \thepage}

\begin{document}

\textit{3.3.2}\\

Suppose $x_n \to q$ as $n \to \infty$. 
By rearranging the secant method, we have $f(x_n) = (x_n-x_{n+1})\left[\frac{f(x_n)-f(x_{n-1})}{x_n-x_{n-1}}\right]$
Taking the limit of this expression gives $\lim_{n \to \infty} f(x_n) = \lim_{n \to \infty} (x_n-x_{n+1})\left[\frac{f(x_n)-f(x_{n-1})}{x_n-x_{n-1}}\right]$.
By the Mean Value Theorem, we know $\lim_{n \to \infty} \left[\frac{f(x_n)-f(x_{n-1})}{x_n-x_{n-1}}\right] = f'(q)$. 
From the problem, we know $f'(q) \neq 0$, so we arrive at $\lim_{n \to \infty} f(x_n) = f(q) = \lim_{n \to \infty} (x_n) - x_{n-1})*f'(q) = (q - q)*f'(q) = 0$.
Thus, q is a zero.

\newline
\textit{3.3.3}\\

Let us begin by finding the Taylor expansions of $f(x + h)$ and $f(x + k)$. 
We find $f(x+h) \approx f(x) + f'(x)*h + /frac{1}{2}f''(x)h^2$.
Likewise, $f(x+k) \approx f(x) + f'(x)*k + /frac{1}{2}f''(x)k^2$.
Rearranging the second expansion gives $\frac{f(x+k) -f(x)-f'(x)k}{k^2} \approx f''(x)$.
We plug this in to the first expansion to find $f(x+h) \approx f(x) + f'(x)*h + \left[\frac{f(x+k) - f(x) - f'(x)*k}{k^2}\right]*h^2$.
Rearranging this gives $f(x + h)*k^2 - f(x)*k^2 - f'(x)*hk^2 - f(x+k)h^2 + f(x)h^2 + f'(x)*h^2k = 0$.

\end{document}
