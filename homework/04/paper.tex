\documentclass{article}
\usepackage[utf8]{inputenc}
\usepackage{amsmath}
\usepackage{amsthm}
\usepackage{amsfonts}
\usepackage{amssymb}
\usepackage{amstext}
\usepackage{gensymb}
\usepackage{graphicx}
\usepackage{enumerate}
%\usepackage{bbold}
%\usepackage{url}
%\usepackage{booktabs}
%\usepackage{marvosym}
%\usepackage{wasysym}
\pagenumbering{arabic}
\usepackage{fancyhdr}
\usepackage[margin=1.0in]{geometry}
\usepackage{eucal}
\def\N{\mathbb{N}}
\def\Z{\mathbb{Z}}
\def\Q{\mathbb{Q}}
\def\R{\mathbb{R}}
\newcommand{\Mod}[1]{\ (\text{mod}\ #1)}
\newcommand{\Problem}[1]{\textbf{Problem #1}}

\pagestyle{fancy}
\fancyhf{}
\rhead{MATH 4500}
\lhead{Alexander Winkles}
\chead{\Large \textbf{Problem Set 4}}
\cfoot{Page \thepage}

\begin{document}

\Problem{6.1.1}\\

For these problems, we will use Lagrange interpolation, where $l_i = \displaystyle\prod_{\substack{j = 0 \\ j \neq i}}^n \frac{x - x_j}{x_i-x_j}$.
\begin{enumerate}[a.]

\item From this, we find $l_0 = \frac{x-7}{10}$ and $l_1 = \frac{x-3}{10}$, so our interpolation becomes 
\begin{equation*}
p_1(x) = 5*\frac{x-7}{10} - \frac{x-3}{10}.
\end{equation*}

\item Using the table, we find $l_0 = \frac{(x-1)(x-2)}{30} = \frac{x^2-3x+2}{30}$, $l_1 = \frac{x^2-9x+14}{6}$, $\frac{x^2-8x+7}{-5}$. 
So we find 
\begin{equation*}
p_2(x) = 146*\frac{x^2-3x+2}{30} + 2*\frac{x^2-4x+14}{6} - \frac{x^2-8x+7}{5}.
\end{equation*}

\item We find $l_0 = \frac{(x-7)(x-1)(x-2)}{(3-7)(3-1)(3-2)}, l_1 = \frac{(x-3)(x-1)(x-2)}{(7-3)(7-1)(7-2)}, l_2 = \frac{(x-3)(x-7)(x-2)}{(1-3)(1-7)(1-2)},$ and $l_3 = \frac{(x-3)(x-7)(x-1)}{(2-1)(2-7)(2-3)}$.
Then
\begin{equation*}
p_3(x) = 10l_0 + 146l_1 + 2l_2 + l_3.
\end{equation*}

\item This problem will have the same $l_i$ for $0 \leq i \leq 3$, but with different $y_i$ values.
So,
\begin{equation*}
p_3(x) = 12l_0 + 146l_1 + 2l_2 + l_3.
\end{equation*}

\item Since most of the y-values are zero, we find that
\begin{equation*}
p_5(x) = \frac{(x-1.5)(x-2.7)(x-3.1)(x+6.6)(x-11.0)}{(-2.1-1.5)(-2.1-2.7)(-2.1-3.1)(-2.1+6.6)(-2.1-11.0)}.
\end{equation*}

\end{enumerate}

\Problem{6.1.2}\\

From Lagrange, we know $p(x) = \sum_{k = 0}^n y_kl_k(x)$. 
From the problem, we know $\exists L:f\mapsto p \Rightarrow Lf = p(x)$ and $Lf = \sum_{k = 0}^n y_kl_k$.
But, $y_k = f(x_k)$, so we have $Lf = \sum_{k = 0}^n f(x_k)l_k$ as desired.
Now let's apply $L$ to $(af + bg)$. 
We then find
\begin{align*}
L(af + bg) &= \displaystyle\sum_{i = 0}^n (af(x_i) + bg(x_i))l_i\\
&= \displaystyle\sum_{i = 0}^n af(x_i)l_i + bg(x_i)l_i\\
&= a\displaystyle\sum_{i = 0}^n f(x_i)l_i + b\displaystyle\sum_{i = 0}^n g(x_i)l_i\\
&= aLf + bLg.
\end{align*}\\

\Problem{6.1.3}\\

Let $Gf = \sum_{i = 0}^n f(x_i)l_i^2$.
From the text, $l_i$ is a polynomial of degree $i$. 
Thus, there is a term $f(x_n)l_n^2$, where $l_n$ is a polynomial of degree n.
Since $l_n$ is squared, its leading term of $ax^n$ will become $a^2x^{2n}$ for some nonzero a. 
Thus, $Gf$ is a polynomail of degree at most 2n.
Each $l_i = \prod_{\substack j = 0, \\ j \neq i}^n \frac{x-x_j}{x_i-x_j}$.
Thus, $l_i(x_j) = 0$ if $i \neq j$ and $l_i(x_i) = 1$ if $i = j$. 
So for any given $x_i$, $Gf(x_i) = f(x_i)$, so it interpolates f at the nodes.
Regardless of what $x_i$ we choose, $(l_i(x_i))^2$ will be positive by the squaring.
Thus, only $f(x_i)$ determines the sign, so $Gf$ is nonnegative so long as f is too.\\

\Problem{6.1.4}\\

Let q be an arbitrary polynomial of degree at most n, and let $p = Lq$, so $p(x) = Lq = \sum_{i = 0}^n q(x_i)l_i$.
Thus, $p(x)$ is a polynomial of degree at most n, and $p(x_i) - q(x_i) = 0$ for $0 \leq i \leq n$.
This implies that $p - q$ is a polynomail of degree at most n with n+1 distinct roots, so $p - q \equiv 0$.
Thus, $p = q$ and $Lq = q$.\\

\Problem{6.1.5}\\

Recall that $Lf = \sum_{i = 0}^n f(x_i)l_i$.
Suppose $f(x_i) = 1$ for $0 \leq i \leq n$. 
Then $q(x) = Lf = \sum_{i = 0}^n f(x_i)l_i = \sum_{i = 0}^n l_i$. 
Since $q(x)$ interpolates $f(x)$, we know $q(x_i) - 1 = 0$. 
This implies $q(x) - 1$, a polynomial of degree at most n, has n+1 zeros.
Therefore, $q(x) - 1 \equiv 0$ for all x and $q(x) = 1$, so $\sum_{i = 0}^n l_i = 1$.\\

\Problem{6.1.9}\\

Suppose g interpolates f at $x_0,x_1,x_2,...,x_{n-1}$ and suppose h interpolates f at $x_1,x_2,x_3,...,x_n$.
Let $f(x_i) = g(x_i) + \frac{x_0 - x_i}{x_n-x_0}[g(x_i) - h(x_i)]$.
Plugging in $x_0$, we find $f(x_0) = g(x_0) + 0$, and since $g(x_0)$ interpolates $f(x_0)$, this is true.
For $1 \leq i \leq n-1$, plugging in $x_i$ results in $f(x_i) = g(x_i) + 0$, since $g(x_i) = h(x_i)$ for the given range, so again this interpolates f.
For $x_n$, we have $f(x_n) = g(x_n) + \frac{x_0 - x_n}{x_n - x_0}[g(x_n) - h(x_n)] = g(x_n) - [g(x_n) - h(x_n)] = h(x_n)$, which interpolates $f(x_n)$, so the given formula interpolates f at all the nodes.\\

\Problem{6.1.10}\\

Let $p(x) = y_0l_0 + y_1l_1 + ... + y_nl_n = \displaystyle\sum_{i = 0}^n y_il_i$, where $l_i = \displaystyle\prod_{\substack{j = 0 \\ j \neq i}}^n \frac{x - x_j}{x_i-x_j}$.
The coefficient for $x_n$ will thus be computed with $\displaystyle\sum_{i = 0}^n y_i\displaystyle\prod_{\substack{j = 0 \\ j \neq i}} \frac{x-x_j}{x_i-x_j}$.
Notice for the numerator in the product, we will obtain a polynomial of degree n who is monic. 
We may thus break this up and focus on $\displaystyle\sum_{i = 0} ^n y_i\displaystyle\prod_{\substack{j = 0 \\ j \neq i}}^n \frac{1}{x_i-x_j}x^n$.
Thus we see $\displaystyle\sum_{i = 0}^n y_i\displaystyle\prod_{\substack{j = 0 \\ j \neq i}}^n (x_i-x_j)^{-1}$ is $x^n$'s coefficient.\\

\Problem{6.1.11}\\

From 6.1.10, we found $\displaystyle\sum_{i = 0}^n y_i\displaystyle\prod_{\substack{j = 0 \\ j \neq i}}^n (x_i-x_j)^{-1}$ to be the coefficient for $x^n$.
For any polynomial q of degree $\neq n-1$, this coefficient must be zero, or it would not be of degree at most n-1. 
Since $q(x_i) = y_i$,\newline $\displaystyle\sum_{i = 0}^n y_i\displaystyle\prod_{\substack{j = 0 \\ j \neq i}}^n (x_i-x_j)^{-1}$ = $\displaystyle\sum_{i = 0}^n q(x_i)\displaystyle\prod_{\substack{j = 0 \\ j \neq i}}^n (x_i-x_j)^{-1}$ = 0, as desired.

\end{document}
