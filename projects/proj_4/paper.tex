\documentclass{article}
\usepackage[utf8]{inputenc}
\usepackage{amsmath}
\usepackage{amsthm}
\usepackage{amsfonts}
\usepackage{amssymb}
\usepackage{amstext}
\usepackage{gensymb}
\usepackage{graphicx}
%\usepackage{bbold}
%\usepackage{url}
%\usepackage{booktabs}
%\usepackage{marvosym}
%\usepackage{wasysym}
\pagenumbering{arabic}
\usepackage{fancyhdr}
\usepackage[margin=1.0in]{geometry}
\usepackage{eucal}

\usepackage{fancyvrb}

\def\N{\mathbb{N}}
\def\Z{\mathbb{Z}}
\def\Q{\mathbb{Q}}
\def\R{\mathbb{R}}
\newcommand{\Mod}[1]{\ (\text{mod}\ #1)}

\pagestyle{fancy}
\fancyhf{}
\rhead{MATH 4500}
\lhead{Alexander Winkles}
\chead{\Large \textbf{Computer Project 4}}
\cfoot{Page \thepage}

\begin{document}

\VerbatimInput{alexander_winkles_lab4.log}

\begin{figure}[h]
\centering
\caption{Duck Approximation}
\includegraphics[scale=0.6]{duck}
\end{figure}

\begin{figure}[h]
\centering
\caption{Runge function with n = 5}
\includegraphics[scale=0.6]{runge_n=5}
\end{figure}

\begin{figure}[h]
\centering
\caption{Runge function with n = 9}
\includegraphics[scale=0.6]{runge_n=9}
\end{figure}

\begin{figure}[h]
\centering
\caption{Runge function with n = 11}
\includegraphics[scale=0.6]{runge_n=11}
\end{figure}

\begin{figure}[h]
\centering
\caption{Runge function with n = 15}
\includegraphics[scale=0.6]{runge_n=15}
\end{figure}

\begin{figure}[h]
\centering
\caption{Spline interpolation of $e^x$}
\includegraphics[scale=0.6]{ex}
\end{figure}

\begin{figure}[h]
\centering
\caption{Comparison of $e^x$ and its spline}
\includegraphics[scale=0.6]{diff}
\end{figure}



\end{document}
