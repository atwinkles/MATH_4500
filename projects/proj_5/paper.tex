\documentclass{article}
\usepackage[utf8]{inputenc}
\usepackage{amsmath}
\usepackage{amsthm}
\usepackage{amsfonts}
\usepackage{amssymb}
\usepackage{amstext}
\usepackage{gensymb}
\usepackage{graphicx}
%\usepackage{bbold}
%\usepackage{url}
%\usepackage{booktabs}
%\usepackage{marvosym}
%\usepackage{wasysym}
\pagenumbering{arabic}
\usepackage{fancyhdr}
\usepackage[margin=1.0in]{geometry}
\usepackage{eucal}

\usepackage{fancyvrb}
\usepackage{graphics}

\def\N{\mathbb{N}}
\def\Z{\mathbb{Z}}
\def\Q{\mathbb{Q}}
\def\R{\mathbb{R}}
\newcommand{\Mod}[1]{\ (\text{mod}\ #1)}

\pagestyle{fancy}
\fancyhf{}
\rhead{MATH 4500}
\lhead{Alexander Winkles}
\chead{\Large \textbf{Computer Project 5}}
\cfoot{Page \thepage}

\begin{document}

\begin{enumerate}
\item We wish to derive the composite Simpson's rule. Let $x_i = a + ih$, $h = \frac{b-a}{2}$ for $0 \leq i \leq n$. Then, we may break the integral as follows: $\int_a^b f(x)dx = \int_{x_0}^{x_1}f(x)dx + \int_{x_1}^{x_2}f(x)dx + ... + \int_{x_{n-1}}^{x_n}f(x)dx$. For each of these, we may approximate the integral with Simpson's rule, as follows:
\begin{align*}
	\int_a^b f(x)dx &= \int_{x_0}^{x_1}f(x)dx + \int_{x_1}^{x_2}f(x)dx + ... + \int_{x_{n-1}}^{x_n}f(x)dx\\
	&\approx \frac{x_1-x_0}{6}\left[f(x_0) + 4f\left(\frac{x_0 + x_1}{2}\right) + f(x_1)\right] + \frac{x_2-x_1}{6}\left[f(x_1) + 4f\left(\frac{x_1 + x_2}{2}\right) + f(x_2) \right] +\\
	& ... + \frac{x_n-x_{n-1}}{6}\left[f(x_{n-1}) + 4f\left(\frac{x_{n-1} + x_n}{2}\right)  + f(x_n)\right]
\end{align*}
Notice that, since we are dividing $[a,b]$ into evenly spaced subintervals, $x_i-x_{i-1} = \frac{x_n-x_0}{n} = h$ where $n$ is the number of subintervals. Thus, we find that 
\begin{equation*}
	\int_a^b f(x)dx \approx \frac{h}{6}\left[f(x_0) + 2\sum_{i=1}^{n-1}f(x_i) + 4\sum_{i=1}^{n}f\left(\frac{x_i+x_{i-1}}{2}\right) + f(x_n)\right]
\end{equation*}
is a composite Simpson's rule that will approximate for $n$ subintervals. Now, let $[a,b] = [a,c]\cup[c,b]$ where $c = \frac{a+b}{2}$. Then, we find that
\begin{align*}
\int_a^b f(x)dx &= \int_a^c f(x)dx + \int_c^b f(x)dx\\	
&\approx \frac{b-a}{12}\left[f(a) + 2f(c) + 4f\left(\frac{a+c}{2}\right) + 4f\left(\frac{c + b}{2}\right) + f(b) \right].
\end{align*}
Now, let $n=4$. Then, with $h = \frac{b-a}{4}$, we find that 
\begin{align*}
\int_a^b f(x)dx &= \int_a^{a+h} f(x)dx + 	\int_{a+h}^{a+2h} f(x)dx + \int_{a+2h}^{a+3h} f(x)dx + \int_{a+3h}^b f(x)dx\\
&\approx \frac{b-a}{24}\left[f(a) + 2\sum_{i=1}^{3} f(a+ih) + 4\sum_{i=1}^4f\left(\frac{a+ih + a + (i-1)h}{2}\right) + f(b)\right].
\end{align*}


\item \fvset{fontsize=\footnotesize}\VerbatimInput{alexander_winkles_proj5_1.log} Thus, errors using this method are small, so the approximation is good. 

\item \fvset{fontsize=\footnotesize}\VerbatimInput{alexander_winkles_proj5_2.log} Thus, errors using this method are small, so the approximation is good. 

\item To create a Romberg formula based on the composite Simpson's rule, we simply substituted using our composite Simpson's rule for the usual trapezoid rule used to compute the $R(n,0)$ values. 

\item \fvset{fontsize=\footnotesize}\VerbatimInput{alexander_winkles_proj5_3.log} Notice with this method that there is a larger error than with just the Romberg. This can be attributed to the error associated with each of the composite Simpson's errors adding up. 

\end{enumerate}
	

	
\end{document}
